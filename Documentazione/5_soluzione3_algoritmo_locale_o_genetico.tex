\section{Il passaggio da un problema di apprendimento ad uno di ricerca}
    Scartato in un primo momento il clustering, il modello di regressione sembrava essere quello più adatto alla risoluzione del problema. Il principale limite, come già menzionato, è l'assenza di valori noti a priori che possano quindi indirizzare il problema verso un corretto uso dell'algoritmo di apprendimento. Mancando la parte vera e propria di "apprendimento" il risultato è una semplice equazione andando fuori dallo scope del progetto, quello di risolvere il problema tramite un algoritmo di Intelligenza Artificiale. 
    
    Rimanendo dentro il tema farebbe molto comodo utilizzare la \textbf{ricerca locale}, in quanto: il nostro modulo deve principalmente restituire dei \textit{giochi simili} all'ultimo gioco gradito dall'utente oppure contrariamente può restituire dei giochi completamente diversi nel caso in cui l'ultimo gioco non è stato particolarmente gradito e implicitamente quindi si potrebbe desiderare un'esperienza diversa.
    
    Definizione di \textit{gioco simile} o \textit{gioco opposto} coincide a meraviglia con i concetti di \textbf{massimo globale} e \textbf{minimo globale} che a seconda di ciò che l'algoritmo desidera trovare corrisponderà al \textbf{ottimo globale}. Per il problema preso in analisi può essere addirittura vantaggioso accontentarsi di \textbf{ottimi locali}, così da restituire non un solo gioco ma un insieme di essi più simili o dissimili all'ultimo giocato.
    
    \section{L'algoritmo da usare}
    \subsection{Algoritmo Genetico Multi Obiettivo}
    Il nostro problema può adattarsi ad un utilizzo di un \textbf{algoritmo genetico}: definire una meta-euristica per stabilire quali individui tra una \textit{popolazione} di giochi (nel nostro caso il backlog) possa essere più vicina all'ultimo gioco giocato dall'utente.
    
    I giochi hanno più di un parametro fondamentale come specificato nella nostra Analisi del problema, quindi la nostra meta-euristica dovrà tener conto di più di un obiettivo da massimizzare. Andiamo quindi incontro ad un \textbf{problema Multi Obiettivo}.
    \paragraph{Approccio ideale e Fronte di Pareto}
    Scartiamo immediatamente un \textit{approccio classico} al problema in quanto non sarebbe fattibile definire una media pesata tra i nostri parametri, trattandosi di stringhe oltre che interi. E' un tipo di operazione molto affine alle soluzioni precedenti, che per gli stessi motivi sono state scartate.
    
    Non rimane quindi che adottare un \textbf{approccio ideale}: si terrà conto dei trade-off di tutte le funzioni obiettivo. L'insieme delle soluzioni non dominate andrà a formare il cosiddetto \textbf{Fronte di Pareto}, una soluzione per noi alquanto comoda perché porta con sè più di una soluzione ottima localmente come inizialmente era previsto.
    
    \section{Problemi}
    Un problema non di poco conto per un'implementazione di un algoritmo genetico arriva nel momento in cui bisogna applicare le operazioni di \textit{selezione} e \textit{mutazione}. Gli individui del nostro problema sono \textbf{preesistenti}, non possono in alcun modo essere generati dei nuovi. Limitare l'algoritmo alla sola operazione di \textit{selezione} sarebbe alquanto riduttivo e di conseguenza anche l'algoritmo genetico non è congruo alla soluzione cercata.
    
    \section{Soluzioni proposte}
    Una soluzione può essere sicuramente quella di regredire ad un "banale" \textbf{algoritmo di ricerca locale} affinchè trovi la soluzione ottima in uno spazio di soluzioni preesistente. Quest'ultimo, guardando attentamente il dataset, è estremamente irregolare. Torna quindi il problema della similarità tra generi e sviluppatori, che essendo stringhe rendono difficile qualsiasi tentativo di stabilire una metrica coerente. 
    
    Pur supponendo di riuscire a stabilire un certo grado di similarità, lo spazio delle soluzioni, per come è strutturato, rende molto difficile la possibilità dell'algoritmo di migliorare iterativamente. Non siamo riusciti a trovare un modo alternativo di formare lo spazio delle soluzioni: ad esempio la rimozione di parametri come genere e/o sviluppatore toglierebbe significato e accuratezza alla soluzione.
    
    \section{Conclusioni}
    In definitiva neanche passando alla ricerca locale non è stato possibile dare una soluzione coerente e soprattuto implementabile. Quello che viene da pensare è che sia un problema che per natura non può essere risolto con l'utilizzo dell'Intelligenza Artificiale o che banalmente lo complicherebbe soltanto.