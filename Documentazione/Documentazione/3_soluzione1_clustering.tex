\section{Motivazioni e risultati attesi}
    Una prima idea era quella di formare raggruppamenti di giochi simili tramite un algoritmo di clustering. Questa soluzione si adattava abbastanza bene inoltre al modulo principale del progetto, in quanto avremmo potuto restituire all'utente giochi casuali di un cluster più o meno grande. Difatti non ci interessa consigliare all'utente il gioco più simile in assoluto agli ultimi giocati, in quanto ciò potrebbe portare una certa noia e ridondanza, e quindi risultati poco graditi.

\section{L'algoritmo da usare}
    L'idea è di usare un algoritmo \textit{partizionale} ed \textit{esclusivo}.
        
    Potrebbe inoltre essere sia \textit{agglomerativo} che \textit{divisivo}, senza troppo impatto sul risultato finale. Tuttavia dobbiamo considerare che questo modulo andrà implementato in un progetto più grande, che prevede di consigliare un gioco a un utente. Possiamo quindi immaginare l'utilità di un algoritmo di tipo \textbf{agglomerativo}: potremmo pensare di creare un circondario dell'elemento atomico che è l'ultimo gioco giocato e fermarci a una certa soglia, quindi non prendendo in considerazione giochi molto distanti. Per lo stesso motivo, sarebbe preferibile un algoritmo \textbf{seriale}; elaborare tutti i pattern contemporaneamente non serve se alcuni di loro non verranno nemmeno considerati. L'unione di queste due considerazioni, unite con il fatto che il dataset di giochi presi in considerazione potrebbe essere di dimensioni considerevoli, potrebbe migliorare di molto le performance del modulo.
        
    Date queste informazioni, e usando come riferimento gli algoritmi messi a disposizione da \texttt{scikit-learn}, un buon candidato è l'algoritmo \texttt{DBSCAN}, in particolare per il fatto che si comporta bene con cluster poco omogenei e non considera gli \textit{outliers}. Questo nel nostro caso è utile, in quanto è possibile nonché probabile che:
    \begin{enumerate}
        \item Ci siano giochi molto generici e quindi appartenenti a cluster più grandi, e giochi facenti parte di nicchie ristrette, appartenenti quindi a cluster minori.
            
        \item Ci siano giochi molto particolari e in definitiva dissimili da qualsiasi altro gioco presente nel dataset.
    \end{enumerate}
        
    \subsection{Problemi}
        L'algoritmo richiede una matrice in cui le colonne rappresentano i parametri e le righe gli elementi. Nel nostro caso ciò ha creato difficoltà, in particolare per quanto riguarda i \textbf{generi}. Dobbiamo considerare che i generi sono collezioni di stringhe, e non valori numerici, da cui nasce la difficoltà nel compararli. Inoltre un singolo elemento non ha senso se preso singolarmente ma ne assume se relazionato ad altri, cosa che ci ha portato alla conclusione che la distanza fra giochi è una misura \textit{semi-metrica}, in quanto la disuguaglianza triangolare non sussiste.
        
        Conseguentemente ai due problemi appena esposti, siamo arrivati alla conclusione che il carico computazionale sarebbe stato piuttosto grande, in quanto si sarebbe dovuta calcolare la distanza fra tutte le coppie di elementi, e che la misura di distanza stessa fosse danneggiata dalla presenza di paragoni fra stringhe non quantificabili in maniera numerica.
        
    \subsection{Soluzioni proposte}
        Analizzando il problema del paragone fra stringhe nel contesto dei generi, abbiamo considerato vari tipi di distanza di edit fra stringhe, ma essendo la misura che interessa a noi semantica piuttosto che testuale, nessuna di queste misure hanno fatto al nostro caso. La soluzione sarebbe, molto più semplicemente, considerare quanti generi hanno in comune i due giochi, rafforzando la conclusione che un singolo gioco preso singolarmente perde di significato.
        
        Abbiamo preso anche in considerazione la possibilità di definire una metrica custom fra due elementi nel contesto di \texttt{scikit-learn}, cosa che non solo si è rivelata difficile, ma ci ha fatto notare che avremmo dovuto semplicemente calcolare la distanza, tramite una metrica custom, per ogni coppia di elementi, problema che sarebbe meglio risolvibile tramite tecniche di regressione, che andremo a discutere nel capitolo successivo.
    
\section{Conclusioni}
    Il nostro problema, in definitiva, non si adatta a essere risolto tramite clustering, in virtù del fatto che i parametri testuali dei nostri elementi assumono un significato e ci danno una misura di similarità solo quando rapportati a parametri testuali di altri elementi.
    
    La definizione di una distanza custom, nonostante sembrasse fattibile all'inizio, introduce una causa di grave inefficienza in quanto dovrebbe essere calcolata fra ogni coppia di elementi. Diverrebbe inoltre un elemento troppo centrale dell'intero algoritmo, sorpassando addirittura i raggruppamenti e rendendoli superflui.