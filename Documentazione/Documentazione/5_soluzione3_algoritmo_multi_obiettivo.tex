\section{Il passaggio da un problema di apprendimento ad uno di ricerca}
    Scartato in un primo momento il clustering, il modello di regressione sembrava essere quello più adatto alla risoluzione del problema. Il principale limite, come già menzionato, è l'assenza di valori noti a priori che possano quindi indirizzare il problema verso un corretto uso dell'algoritmo di apprendimento. Mancando la parte vera e propria di "apprendimento" il risultato è una semplice equazione andando fuori dallo scope del progetto, quello di risolvere il problema tramite un algoritmo di Intelligenza Artificiale. 
    
    Rimanendo dentro il tema farebbe molto comodo utilizzare la ricerca locale, in quanto: il nostro modulo deve principalmente restituire dei \textit{giochi simili} all'ultimo gioco gradito dall'utente oppure contrariamente può restituire dei giochi completamente diversi nel caso in cui l'ultimo gioco non è stato particolarmente gradito e implicitamente quindi si potrebbe desiderare un'esperienza diversa.
    
    Definizione di \textit{gioco simile} o \textit{gioco opposto} coincide a meraviglia con i concetti di \textbf{massimo globale} e \textbf{minimo globale} che a seconda di ciò che l'algoritmo desidera trovare corrisponderà al \textbf{ottimo globale}.Per il problema preso in analisi può essere addirittura vantaggioso accontentarsi di \textbf{ottimi locali}, così da restituire non un solo gioco ma un insieme di essi più simili o dissimili all'ultimo giocato.
    
    \section{L'algoritmo da usare}