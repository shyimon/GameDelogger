    Avendo un certo grado di incertezza sull'algoritmo da impiegare per la risoluzione di questo problema, abbiamo deciso di lavorare innanzitutto sulla definizione dello stesso e sulla raccolta dei dati. Queste sono fasi fondamentali, e comprendere a fondo il problema con cui abbiamo a che fare è un passo indispensabile che prescinde dall'algoritmo usato.

\section{Specifica PEAS}
    \subsubsection{Performance}
        La misura di performance dell'agente è direttamente legata alla percezione che un utente avrebbe dei giochi considerati dal sottomodulo come simili, introducendo quindi un certo grado di soggettività. Possiamo pensare di implementare ciò tramite \textbf{reinforcement learning}. Nonostante ciò sia difficile all'interno del sotto-modulo, possiamo pensare all'interazione con il sistema in cui esso verrà implementato; se un utente recensisce positivamente un gioco consigliato da GameDelogger sulla piattaforma ospite, questo varrà come feedback positivo, e viceversa. Questa soluzione resta fuori dallo scope del progetto in quanto servirebbe un feedback costante da parte di un bacino d'utenza di cui non disponiamo. In maniera simile, nemmeno un dataset siffatto non esiste.
        
    \subsubsection{Environment}
        L'ambiente in cui l'agente opera è composto dai giochi disponibili e tutte le informazioni relative a essi che abbiamo a disposizione. L'ambiente è:
        \begin{itemize}
            \item \textbf{Completamente osservabile}, in quanto abbiamo accesso a tutte le informazioni relative a ogni singolo titolo in qualsiasi momento.
            \item \textbf{Deterministico}, in quanto i cluster da individuare dipendono da caratteristiche del gioco definite a priori e non soggette ad alcun tipo di casualità.
            \item \textbf{Sequenziale}, in quanto le azioni dell'agente hanno potenzialità di influenzare le scelte dell'utente, che a loro volta creeranno uno stato diverso per l'agente.
            \item \textbf{Statico}, in quanto l'agente considera il backlog come una lista di elementi non soggetta a mutazioni nel mentre delibera.
            \item \textbf{Discreto}, in quanto abbiamo un numero discreto di possibili giochi da consigliare.
            \item A \textbf{singolo agente}, in quanto la presenza di più agenti sarebbe superflua.
        \end{itemize}
        
    \subsubsection{Actuators}
        Creazione e restituzione di uno o più giochi ritenuti rilevanti.
        
    \subsubsection{Sensors}
        I sensori dell'agente gli permettono di ottenere tutte le informazioni relative al backlog e ai giochi in esso contenuti.
        
\section{Analisi del problema}
    Dopo aver analizzato l'ambiente, ci rendiamo conto che è molto importante definire con precisione l'entità del \textbf{gioco}, per metterlo in qualche tipo di relazione con gli altri giochi.
    
    \subsection{Il gioco}
        Un oggetto \texttt{gioco} identifica univocamente un prodotto videoludico, in questo specifico caso una entry del backlog, ed è composto dai seguenti parametri:
        \begin{itemize}
            \item \textbf{Nome}: è una stringa che identifica il gioco. Seppur raro, due giochi diversi potrebbero avere lo stesso nome, quindi non verrà utilizzata come identificativo.
            
            \item \textbf{Tempo di completamento}: il tempo di completamento medio, ottenibile tramite un semplice tool di web scraping da un sito come \url{www.howlongtobeat.com}.
          
            \item \textbf{Genere}: definito come stringhe separate da una virgola, il genere o generi del gioco è un fattore notoriamente importante per definirne la similarità rispetto ad altri giochi.
           
            \item \textbf{Sviluppatore}: è in buona sostanza l'autore o gli autori del gioco. Anche esso definito come stringa. Possiamo facilmente immaginare la rilevanza di questo dato; un autore tende ad avere un'impronta artistica rilevante su tutti i giochi che crea, aumentandone il grado di similarità anche se tutti gli altri parametri previamente descritti sono poco affini.
            
            \item \textbf{Anno di pubblicazione}: banalmente l'anno della prima pubblicazione del titolo che può essere utile per definire la similarità tra giochi. Quest'ultimi se usciti intorno allo stesso periodo possono avere molto di più in comune grazie alle tecnologie utilizzate o all'influenza di alcuni giochi più significativi dell'epoca, per questo il fattore temporale può essere rilevante.
        \end{itemize}
        
        Questi dati devono chiaramente essere estratti e organizzati, operazioni che andiamo di seguito a documentare.
        
\section{Dataset}
    \subsection{Scelta e ottenimento del dataset}
        L'idea è quella di raccogliere le informazioni di un certo numero di giochi dal sito \url{www.howlongtobeat.com}. Per fare ciò ci siamo avvalsi dell'utilizzo di un estensione browser per effettuare il web scraping del sito web precedentemente citato, ricavando da esso gli attributi necessari per il nostro scopo, quali: il nome del gioco, il tempo di completamento, il/i genere/i, la casa di sviluppo e l'anno di pubblicazione. Da questa operazione abbiamo ricavato un dataset semi-strutturato di 200 titoli.
        
        Esso sarà quindi in formato CVS e avrà una tale struttura:
        \begin{table}[h]
            \begin{tabular}{|c|c|c|c|c|}
            \hline
            \textbf{Nome} & \textbf{Tempo} & \textbf{Genere}               & \textbf{Developer} & \textbf{Anno} \\ \hline
            Bioshock: Infinite & 15½ Hours & First Person, Shooter, Action & Irrational Games & 2013\\ \hline
            Doom               & 6½ Hours       & First Person, Shooter         & id Software & 1993        \\ \hline
            Final Fantasy VII  & 51 Hours       & Role-Playing                  & Square & 1997             \\ \hline
            \end{tabular}
        \end{table}
        
    \subsection{Data preparation}
    
        \subsubsection{Pre-processing}
            L'unica operazione di pre-processing da fare è sul tempo: esso è espresso nel formato "numero di ore + $\frac{1}{2}$ (opzionale) + Hours", mentre a noi interessa un formato del tipo "numero di ore + .5(opzionale)". È Solo necessario, quindi, rimuovere la stringa "Hours" e trasformare $\frac{1}{2}$ in .5 quando presente.
        
        \subsubsection{Data Cleaning}
            Per una piccola quantità di giochi nel sito manca il genere o anche se c'è non è stato rilevato dallo scraping. Stesso discorso per il tempo, dato che alcuni titoli presenti sul sito avevano a disposizione solo una categoria di tempo, essa non veniva rilevata dallo scraper che prendeva solo una tipologia specifica di tempo. Vi sono varie possibili soluzioni per riempire le celle mancanti, quali: eseguire una ricerca incrociata con altri siti come per esempio lo stesso Wikipedia, il quale riporta il genere di molti giochi; intervenire umanamente ricontrollando \url{www.howlongtobeat.com}, o se si è a conoscenza del titolo aggiungerlo autonomamente.
            
        \subsubsection{Feature Scaling}
           l'anno di rilascio verrà normalizzato secondo la \textbf{min-max normalization}. Questo tipo di normalizzazione è stato scelto perché inquadra solo il periodo di tempo in cui un gioco è stato rilasciato, senza considerarne la distribuzione, dato che non ci interessa, come per esempio farebbe la \textit{z-score normalization}. Alcuni esempi di questo tipo di normalizzazione sono riportati di seguito.
            \begin{center}
                \begin{tabular}{|c|c|}
                \hline
                $x$  & $x_{norm}$ \\ \hline
                1985 & 0.05     \\ \hline
                1997 & 0.36     \\ \hline
                2006 & 0.59     \\ \hline
                2012 & 0.74     \\ \hline
                2022 & 1.00     \\ \hline
                \end{tabular}
            \end{center}
        
            \newpage
            Se per l'anno di rilascio la distribuzione non ci interessa, per la durata è importante, in quanto descrive tendenze di mercato rilevanti. Del resto, un gioco è considerato "lungo" o "corto" anche in base alla durata media degli altri giochi. Per questo motivo abbiamo scelto la \textit{z-score normalization}, di cui riportiamo alcuni valori di seguito. Sul dataset ottenuto tramite web scraping abbiamo calcolato un valore della media della distribuzione $\bar{x} = 36.32$ e un valore di distribuzione standard $\sigma = 51.43$.
            \begin{center}
                \begin{tabular}{|c|c|}
                \hline
                $x$  & $x_{znorm}$ \\ \hline
                10 & -0.51176     \\ \hline
                15.5 & -0.40482     \\ \hline
                22 & -0.27844     \\ \hline
                51.5 & 0.28544     \\ \hline
                80 & 0.84931     \\ \hline
                \end{tabular}
            \end{center}
            
        \subsubsection{Feature Selection}
            La selezione dei dati appropriati è stata già discussa in precedenza, ma andiamo ora a motivare ed elaborare il perché di queste scelte.
            
            Ecco dunque le caratteristiche da noi scelte o costruite e il perché le riteniamo utili:
            \begin{itemize}
                \item \textbf{Nome:} Non ha rilevanza per quanto riguarda l'algoritmo, ma è utile per mostrare a video i nostri risultati.
                
                \item \textbf{Tempo di completamento:} Nonostante non sia una feature determinante della similarità fra giochi, molti utenti tendono a preferire giochi di durata simile, sia per gusto personale che per quantità di tempo a disposizione. Un'altra feature che potrebbe essere importante è per esempio la durata media di una sessione di gioco, a prescindere dalla durata dello stesso. Nonostante la questione sia interessante, resta aperta a causa di una mancanza totale di un tale dataset o API che permettano di reperire tali informazioni.
                
                \item \textbf{Genere:} È il metro di similarità più importante per noi, essendo quello che descrive meglio il contenuto del gioco. Un'analisi di mercato sarebbe utile a definire la similarità fra generi, ma per ora ci siamo limitati a definirla in base al numero di generi in comune fra i due giochi.
                
                \item \textbf{Developer:} Abbiamo preso in considerazione l'impronta artistica e stilistica che il team di sviluppo ha su un determinato gioco come dato importante.
                
                \item \textbf{Anno di pubblicazione:} Questo è un altro dato che non ci dice molto sul contenuto del gioco, ma che abbiamo ritenuto importante a causa dei grandi cambiamenti che il mercato videoludico ha subito dalla sua nascita. Gli avanzamenti tecnologici hanno portato a un importante cambiamento stilistico nello sviluppo di videogiochi, e le persone tendono a identificarsi con una determinata era videoludica che hanno particolarmente a cuore, motivo per cui abbiamo ritenuto questo dato rilevante.
            \end{itemize}
            
            Andiamo ora a vedere alcuni dei dati che non abbiamo preso in considerazione nonostante fossero presenti su \url{www.hoelongtobeat.com}, il perché e il se potrebbero essere usati in sviluppi futuri del progetto:
            \begin{itemize}
                \item \textbf{Piattaforme di pubblicazione:} In passato la piattaforma di appartenenza di un gioco era rilevante, in quanto diverse compagnie implementavano diverse politiche e addirittura diverse tecnologie per le loro console. Tuttavia oggi queste differenze si sono appiattite molto e tutte le console hanno tecnologie e prestazioni molto simili. Forse più importante, questo modulo è pensato per essere implementato su un servizio come Playstation Now, Microsoft GamePass, Steam, etc. Chiaramente in questi casi la piattaforma è implicata, e sarebbe informazione ridondante.
                
                \item \textbf{Publisher:} Questa è la casa di pubblicazione di un videogioco. Molto raramente i giochi pubblicati dalla stessa casa hanno similarità stilistiche, in quanto queste sono definite piuttosto dal team di sviluppo.
                
                \item \textbf{Altri tempi di completamento:} Il sito da noi preso in considerazione ha tempi di completamento per la campagna principale, la campagna + gli extra (quello da noi preso in considerazione) e il completamento del gioco al 100\%. Le altre due categorie potrebbero essere molto utili per alcune piattaforme come Steam o Xbox, o qualsiasi altro servizio che fornisca la percentuale di completamento di un gioco per un determinato utente. Possiamo immaginare infatti di dare più peso al tempo di completamento del gioco al 100\% se notiamo che l'utente tende a completare molti giochi al 100\%.
            \end{itemize}
            
        \subsubsection{Data Balancing}
            Siccome ogni gioco è unico, non abbiamo riscontrato necessità di eseguire un bilanciamento dei dati su questo particolare dataset. Inoltre c'è da dire che \textbf{ci aspettiamo} un certo sbilanciamento in un problema del genere, cosa notata durante lo studio di fattibilità.