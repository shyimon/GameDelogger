\section{Il perché della non riuscita del progetto}
    Dopo un'attenta analisi del percorso di sviluppo, delle soluzioni proposte e dei dati che abbiamo a disposizione, abbiamo individuato alcune imperfezioni intrinseche del nostro problema, che lo rende poco adatto ad essere risolto con le tecniche a disposizione. In particolare:
    \begin{itemize}
        \item L'elemento di apprendimento del progetto si basa sul feedback di ogni singolo utente, allo scopo di personalizzare l'esperienza. Questo, oltre a rimandare al \textit{reinforcement learning}, richiede una base di utenza di cui non disponiamo.
        
        \item In funzione del punto precedente abbiamo preso in considerazione la similarità fra giochi, parametro che non necessariamente richiede feedback utente, ma così facendo siamo stati vittime di un \textit{bias} nei confronti dell'intelligenza artificiale: abbiamo infatti dato per scontato che questa fosse la soluzione migliore quando invece non fa altro che complicare il problema.
        
        \item Il nostro dataset contiene valori testuali che vanno paragonati sotto il punto di vista semantico (in particolare la similarità fra generi videoludici), cosa per cui sarebbe utile un'analisi di mercato o un dataset che al momento non esiste. La proposta di scartare questa colonna del dataset è infattibile, in quanto è la più importante per definire la similarità fra giochi.
    \end{itemize}
    
    Una volta delineati i difetti del nostro progetto, ci proponiamo di cercare possibili soluzioni o sviluppi futuri.
    
    \section{Soluzioni che non fanno uso di Intelligenza Artificiale}
        Come abbiamo già detto in precedenza, la similarità fra giochi non deve necessariamente essere definita tramite intelligenza artificiale e anzi, una semplice equazione sarebbe più che sufficiente. Possiamo immaginarla in una forma simile a quella descritta di seguito.
        
        Descriviamo innanzitutto i quattro parametri di similarità, ossia in base alla \textbf{durata}, all'\textbf{anno di uscita}, ai \textbf{generi} e al \textbf{team di sviluppo}.
        
        \subsubsection{Similarità durata}
            Definita come $simD(gioco_1, gioco_2) = \frac{min(durataGioco_1, durataGioco_2)}{max(durataGioco_1, durataGioco_2)}$. Questo valore sarà al massimo 1, nel caso di durata identica.
            
        \subsubsection{Similarità anno di pubblicazione}
            Definita come $simA(gioco_1, gioco_2) = \frac{min(annoGioco_1, annoGioco_2) - minAnno}{max(annoGioco_1, annoGioco_2) - minAnno}$. Anche questo valore varia da 0 a 1. C'è da specificare che il parametro $minAnno$ corrisponde all'anno di pubblicazione più remoto registrato nel dataset al momento.
            
        \subsubsection{Similarità generi}
            Definita come $simG(gioco_1, gioco_2) = \frac{numGeneriInComune}{max(numGeneriGioco_1, numGeneriGioco_2)}$. Ancora una volta, questo valore sarà al massimo 1, nel caso in cui i due giochi siano descritti con lo stesso numero di generi, tutti uguali.
            
        \subsubsection{Similarità sviluppatori}
            Definita in maniera booleana come $simS$. Vale 1 se i due giochi hanno lo stesso sviluppatore o team di sviluppo, 0 altrimenti.
        
        \subsubsection{Similarità giochi}
            Usando le similarità descritte prima, definiamo infine la similarità fra due giochi come segue:
            \begin{center}
                \begin{equation*}
                    simGiochi = w(simD) + x(simA) + y(simG) + z(simS)
                \end{equation*}
            \end{center}
            
            Se vogliamo un valore di similarità compreso fra 0 e 1 ci basta porre $w + x + y + z = 1$. Ciò che ci resta ora è una semplice somma pesata, dove i pesi possono essere visti come l'importanza che diamo a ognuna delle caratteristiche rispetto alle altre.
            
            Questi pesi potrebbero essere decisi tramite un'analisi di mercato, dando una soluzione definitiva e immutabile al problema che ci siamo posti all'inizio del progetto: il valore di similarità verrebbe usato per restituire all'utente una selezione di giochi simili a quelli che ha maggiormente apprezzato.
            
    \section{Soluzioni che fanno uso di intelligenza artificiale}
        Abbiamo stabilito che la definizione di similarità fra giochi non necessita di una soluzione basata su Intelligenza Artificiale; tuttavia, possiamo pensare di convertire questo valore di similarità in un valore di \textit{affinità} personalizzato per ogni utente, e in queste quattro variabili potrebbero essere dettate, piuttosto che da un'analisi di mercato, da un grado di interesse da parte dell'utente: se esso tende a mostrare poco interesse per la durata (quindi se $simD$ varia molto da un titolo gradito all'altro), il suo peso sarà minore, andando a favorire parametri per cui ha un interesse più specifico (per continuare l'esempio, se gradisce solo giochi usciti dopo il 2015, il valore $simA$ avrà un gran peso e influenzerà molto il risultato).